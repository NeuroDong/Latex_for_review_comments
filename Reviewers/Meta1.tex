% Response to Editor
\AssociateEditor
\begin{generalcomment}
The paper presents an interesting and relevant survey, and it has received good reviews. However, I see important deficiencies in the presentation and clarity of the work, and also some missing experiments. All of them should considered when preparing a new revised version of the manuscript.
\end{generalcomment}
\begin{revmeta}[]
We would like to extend my sincere gratitude for your thorough review and valuable feedback on our manuscript. We appreciate the time and effort you have invested in evaluating our work and fully acknowledge the deficiencies you have identified in the presentation and clarity of our study, as well as the missing experiments.

We understand the importance of addressing these issues to enhance the quality and impact of our research. Below, we outline the steps to address each of your concerns.
\end{revmeta}

\begin{revcommentToAssociateAuthor}
- The authors should perform some experiments to provide better guidelines and insights to the readership.
\end{revcommentToAssociateAuthor}
\begin{revmeta}[]
Thank you very much for your valuable comments and suggestions. Our initial manuscript included no experiments because we followed other high-quality surveys \cite{wu2020comprehensive,ji2021survey,li2021survey,zhang2023survey,li2021low} that did not include experiments. However, in light of your valuable feedback, we conducted some necessary experiments to provide better guidelines and insights to the readership. In the revised manuscript, we added a new section (Section VII on the 17th page of the revised manuscript) dedicated to comparative experiments, and we anonymously open-sourced the code implementing all experiments. For your convenience, we have restated the experimental part of the revised manuscript below.
\end{revmeta}

\clearpage
\printbibliography[heading=bibintoc, heading=bibliography, title={References}, section=\therefsection]
