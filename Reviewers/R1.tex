% Reviewer 1

\reviewer

\begin{generalcomment}
The paper offers an extensive overview of the latest confidence calibration techniques for deep imbalanced learning, presenting significant and relevant research. However, there are a few areas where improvement could be made.
\end{generalcomment}
\begin{revmeta}[]
We would like to express our sincere gratitude for your feedback and valuable suggestions on our manuscript. We have carefully considered each of your points and revised them carefully.
\end{revmeta}

\begin{revcomment}
- A summary of the practical domains where these methods have been applied, or could be applied, is missing.
\end{revcomment}
\begin{revresponse}[]
Thank you for the pertinent suggestions. We agree with your point of view. We added an application section (Section VIII) on the 18th page of the revised manuscript. For your convenience, we put the newly added application section below, and we hope it will satisfy you.
	
\begin{changes}
\lipsum[1]

\begin{lstlisting}[language=Python, caption={Algorithm},frame=single]
	for i in range(1, N+1):
	# Run
	perform_operation(i)
\end{lstlisting}

\lipsum[2]

\centering
\includegraphics[width=0.5\textwidth,keepaspectratio]{imgs/IU1.jpg}
\captionof{figure}{IU}
\end{changes}
	
\end{revresponse}

\clearpage
\printbibliography[heading=bibintoc, heading=bibliography, title={References}, section=\therefsection]